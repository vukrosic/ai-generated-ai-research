\documentclass[11pt,letterpaper]{article}

\usepackage[utf8]{inputenc}
\usepackage{amsmath}
\usepackage{amsfonts}
\usepackage{amssymb}
\usepackage{graphicx}
\usepackage{booktabs}
\usepackage[margin=1in]{geometry}

\title{Neural Network Architectures for Polynomial Curve Fitting: A Comparative Study}
\author{Test Author}
\date{\today}

\begin{document}

\maketitle

\begin{abstract}
This paper presents a comprehensive empirical study of neural network architectures and optimization techniques for polynomial curve fitting tasks. We systematically evaluate 3 different network architectures combined with 3 optimization algorithms across polynomial curves of varying degrees. Our experimental evaluation comprises 3 completed experiments, providing insights into the effectiveness of different model-optimizer combinations for curve fitting tasks.
\end{abstract}

\section{Introduction}

Polynomial curve fitting represents a fundamental problem in machine learning and computational mathematics. This paper addresses the systematic comparison of neural network architectures and optimization techniques for polynomial curve fitting through comprehensive experimental evaluation.

\section{Methodology}

We evaluate three classes of neural network architectures: linear models, shallow networks, and deep networks. Each architecture is combined with different optimization algorithms including SGD, Adam, RMSprop, and AdaGrad.

\section{Results and Analysis}

Table~\ref{tab:performance} presents the performance of different model configurations.

\begin{table}[h]
\centering
\caption{Model performance comparison}
\label{tab:performance}
\begin{tabular}{@{}lllrrr@{}}
\toprule
Architecture & Optimizer & Degree & Val Loss & Train Loss & Time (s) \\
\midrule
linear & sgd & 2 & 0.002000 & 0.001000 & 20.00 \\
shallow & adam & 3 & 0.003000 & 0.001500 & 35.00 \\
deep & rmsprop & 4 & 0.004000 & 0.002000 & 50.00 \\
\bottomrule
\end{tabular}
\end{table}

\section{Discussion}

Our experimental evaluation reveals significant performance variations across different architecture-optimizer combinations. The results provide practical guidance for curve fitting applications.

\section{Conclusion}

This study provides comprehensive benchmarks for neural network curve fitting and demonstrates the importance of architecture and optimizer selection for optimal performance.

\end{document}
